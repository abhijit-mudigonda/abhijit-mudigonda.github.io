Throughout this post, I'll assume familiarity with the basic notions of Hodge
theory. Here's a writeup for anyone curious. 


Let $X$ be a smooth and proper complex variety. The various cohomologies of $X$ encapsulates
important structural information of $X$ in the form of abelian groups and linear algebra.
For example, if $X$ is defined over $\Q$, the action of the absolute Galois group
$G_\Q$ on $X$ induces a representation of $G_\Q$ on the (\'{e}tale cohomology) of $X$. 
Any subvariety induces a cohomology class -- its fundamental class as a submanifold. 
A natural question is then whether we can go backwards, associating subvarieties to 
cohomology classes. If so, we could translate objects and maps from the "linearized" setting
to subvarieties and maps between them. This question is addressed by the Hodge conjecture
and the Tate conjecture.


Let $\mathrm{Hdg}^i(X) \subset H^{2i}_{\t{sing}}(X;\Q)$ denote the \textbf{$\Q$-vector space}
of \textbf{Hodge classes} -- $\Q$-linear combinations of the fundamental classes of 
$i$-codimensional complex subvarieties of $X$. \anote{TODO why it's in i,i}

\begin{conjecture}[Hodge]
  \begin{equation}
    \label{eq:1}
    \mathrm{Hdg}^i(X) = H^{2i}_{\t{sing}}(X;\Q) \cap H^{i,i}(X;\C)
  \end{equation}
\end{conjecture}

Thus, every cohomology class which "could" arise from a linear combination
of subvarieties does so. 

\anote{say more here}


I hope to say some more about the Hodge and Tate conjectures in a subsequent post, but for
now I wanted to post about a computation that Professors Madhav Nori and Frank Calegari
independently assigned to me as an exercise last fall: verify the Hodge conjectures
when $X$ is a product of elliptic curves, say
$X = E_1 \times E_2 = \C/\Lambda_1 \times \C/\Lambda_2$.


We have
\[H^{1,1}(X;\C) = \C\angs{dz_1 \wedge d\overline{z_1}, dz_1 \wedge d\overline{z_2}, dz_2 \wedge d\overline{z_1}, dz_2 \wedge d\overline{z_2}} \cong \C^4.\]

Let's compute a basis for $H^2(X;\Q)$ with respect to the basis 
$\{dz_1, d\overline{z_1}, dz_2, d\overline{z_2}\}$. 

By the K\"unneth formula, 
\[H^2(X;\Q) \cong H^2(E_1;\Q) \otimes H^0(E_2;\Q) \oplus H^1(E_1;\Q) \otimes H^1(E_2;\Q) \oplus H^0(E_1;\Q) \otimes H^2(E_2;\Q) \cong \Q^6.\]
The one-dimensional contributions from $H^2(E_1;\Q) \otimes H^0(E_2;\Q)$ and $H^0(E_1;\Q) \otimes H^2(E_2;\Q)$
are exactly the fundamental classes of subvarieties $O_1 \times E_2$ and $E_1 \times O_2$,
where $O_i$ is the point at infinity of $E_i$ (the point at infinity
is a rational point so these are indeed subvarieties).

\begin{remark}
  It is not true that distinct subvarieties give rise to 
  distinct cohomology classes. For example, for every rational point
  $P \in E_1$, the cohomology class associated to $P \times E_2$ is the same.
  \anote{check this?}
\end{remark}

It remains to study $H^1(E_1;\Q) \otimes H^1(E_2;\Q)$ and its intersection
with $H^{1,1}(X;\C)$. 


Without loss of generality, let $\Lambda_i := \Z\angs{1,\tau_i} \subset \C$ -- 
dilating the lattice does not affect the curve up to isomorphism. 
Then, the line from $0$ to $1$ and the line from $0$ to $\tau_i$
form a basis for $H_1(\C/\Lambda_i;\Q)$. Call these $l$ and $l'$. 
By the universal coefficients theorem, 
\[H^1(\C/\Lambda_i;\Q) \cong \Hom\paren{H_1(\C/\Lambda_i;\Q),\Q},\]
and a basis is given by forms $\omega, \omega'$ such that 
\[\int_l \omega = \int_{l'} \omega' = 1, \int_{l'} \omega = \int_l \omega' = 0.\]


We can check that 
\[\













